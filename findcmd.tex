\documentclass{article}
\usepackage{CJK}
\linespread{1.5}
\setlength{\parindent}{2em}
\setlength{\parskip}{1em}
\usepackage[CJKbookmarks=true]{hyperref}
\begin{document}
\begin{CJK*}{UTF8}{gbsn}
\tableofcontents{}
\section{名称}
find - 在目录层级中搜索文件
\section{摘要}
\verb|find [-H] [-L] [-P] [-D debugopts] [-Olevel] [path...] [expression]|
\section{描述}
这份手册页描述了GNU版本的\textbf{find}。GNU \textbf{find}根据优先级规则,从左到右地计算表达式,然后在目录树中搜索每一个指定的文件名,直到输出是已知的,接下来find转向处理下一个文件名。

如果你在一个安全因素很重要的环境中使用\textbf{find},你应该读读查找工具的文档中的``安全考虑''章节,它可能被称为\textbf{寻找文件}并且带有一些查找工具。那份文档也包含了许多比这份手册页更进一步的细节和讨论,因此你可能觉得那是一份更有用的信息源。
\section{选项}
\textbf{-H}、\textbf{-L}和\textbf{-P}选项控制对待符号链接的方式。跟在这些后面的所有命令行参数被视为文件或目录的名字以被测试,除非遇到以`-'开头的参数,或者参数`('和`!'。这些参数以及参数后面的所有东西都被视为用来描述什么该搜索的表达式。如果没有指定路径,就从当前目录开始。如果没有给出表达式,那么表达式\verb|-print|将被使用。

这份手册页谈论表达式列表中的`选项'。这些选项控制\textbf{find}的行为都直接在最后的路径名之后被指定。如果有的话,五个`真正'的选项\textbf{-H、-L、-P、-D}和\textbf{-O}必须出现在第一个路径名之前。双破折号同样可以用来表示任何不是选项的剩余参数。
\begin{description}
\item[-P] 永远不要跟随符号链接。这是默认的行为。当\textbf{find}测试或者打印一个文件的信息时,并且这个文件是一个符号链接,那么所使用的信息应该是从这个符号链接本身所得到的属性。
\item[-L] 跟随符号链接。当\textbf{find}测试或者打印关于文件的信息时,所使用的信息应该从链接所指向的文件中取得,而不是链接本身。使用这个选项意味着\verb|-noleaf|。如果你稍后使用了\verb|-P|选项,\verb|-noleaf|仍会起作用。如果\verb|-L|正在实行并且\textbf{find}在搜索过程中遇到了一个指向子目录的符号链接,那么由符号链接所指向的那个子目录将被搜索。

当\verb|-L|选项在起作用的时候,谓词\verb|-type|总是去匹配由符号链接所指向的文件的类型而不是链接本身。使用\verb|-L|导致谓词\verb|-lname|和\verb|-ilname|总是返回假。
\item[-H] 不要跟随符号链接,除非是在处理命令行参数。当\textbf{find}测试或者打印关于文件的信息时,这些信息应该取自符号链接自身的属性。唯一的例外是当命令行中指定的一个文件是一个符号链接并且链接可以被回溯的时候。对于这种情况,信息来自于链接所指向的任何东西。链接本身的信息作为后备,在被符号链接指向的文件不能被测试时使用。如果\verb|-H|正在使用并且命令行中指定的其中一个路径是一个指向目录的符号链接,那么目录的内容将被测试。

如果在\textbf{-H、-L}和\textbf{-P}中多于一个被指定,每一个都会覆盖其余的那些;在命令行中出现的最后一个将起作用。由于\verb|-P|是默认选项,因此它应该被认为正在起作用除非\verb|-H|或者\verb|-L|中的一个被指定了。

GNU的\textbf{find}在开始任何搜索之前,在处理命令行的过程中自己统计文件。这些选项也影响了那些参数怎么被处理。特别地,有很多测试用来比较列在命令行中的文件和当前考虑中的文件。在每一个例子中,命令行所指定的文件都会被测试并且它的一些属性会被保存。如果指名的文件确实是个符号链接,并且启用了\textbf{-P}选项,那么用于比较的信息将从符号链接取得。否则,信息取自链接所指的文件的属性。如果\textbf{find}不能跟随链接,那么链接本身的属性将被使用。

当\textbf{-H}或者\textbf{-L 选项正在使用时},作为\textbf{-newer}的参数的\textbf{任何列出来的符号链接}都会被反引用,并且时间戳会从符号链接所指向的文件取得。同样地,\textbf{-newerXY}、\textbf{-anewer}和\textbf{-cnewer}选项也需要这么考虑。

选项\textbf{-follow}和\textbf{-L}有着同样的效果,尽管它只在它出现的地方起作用。
\item[-D] 调试选项\\
打印诊断信息;这在诊断为什么\textbf{find}为什么没有做你希望的事情的时候很有用。列表中的调试选项应该以逗号分隔。释放出来的查找工具之间对调试选项的兼容性并没有保证。如果需要完整的有效调试选项的列表,参见\textbf{find -D help}的输出结果。有效的调试信息包括
\begin{description}
\item[help] 解释调试选项
\item[tree] 以原始和优化过的形式展示表达式树。
\item[stat] 在文件被系统调用\textbf{stat}和\textbf{lstat}测试时打印消息。\textbf{find}程序会尽可能地减少这种调用。
\item[opt] 打印和表达式树的优化有关的诊断信息;参阅\verb|-O|选项。
\item[rates] 打印总结表明每一个谓词成功或者失败的频率。
\end{description}
\item[-Olevel] 启用请求的最优化。\textbf{find}程序的读取器在保留全部的效果时尝试加快执行;就是说,带有副作用的为此不会被重新排序到有关的每一个上去。
\end{description}
\end{CJK*}
\end{document}
